\chapter{Scoping}

\section{Primo Project Scoping Meeting}

\subsection{User Stories}

Nel corso del primo incontro di scoping, il team di progetto CityTwin ha identificato e definito le user stories chiave. Questa fase è stata fondamentale per ottenere una comprensione dettagliata delle esigenze degli utilizzatori finali e ha fornito una guida iniziale per lo sviluppo del sistema di digital twin.

\subsubsection{User Story 1: Visualizzazione dello Stato del Sistema tramite GUI}

\textbf{Come} utente interessato al monitoraggio del sistema CityTwin \\
\textbf{Voglio} accedere a una GUI (Control Panel) \\
\textbf{In modo da} visualizzare chiaramente lo stato attuale dei nodi Mainstay e Resource del sistema.

\textbf{Criteri di Accettazione:}
\begin{itemize}
  \item La GUI deve fornire un'interfaccia intuitiva e facile da comprendere.
  \item Lo stato di ciascun nodo (Mainstay e Resource) deve essere chiaramente indicato.
  \item L'utente deve essere in grado di distinguere facilmente i nodi online da quelli offline.
\end{itemize}

\subsubsection{User Story 2: Visualizzazione delle Risorse nella Città tramite GUI}

\textbf{Come} utente interessato alla localizzazione delle risorse nel sistema CityTwin \\
\textbf{Voglio} utilizzare la GUI \\
\textbf{In modo da} visualizzare la posizione delle risorse nella città, con indicazione del nome e dello stato (online/offline).

\textbf{Criteri di Accettazione:}
\begin{itemize}
  \item La mappa nella GUI deve mostrare chiaramente la posizione delle risorse.
  \item Ogni risorsa deve essere contrassegnata con il suo nome e stato di funzionamento.
  \item L'aggiornamento della posizione delle risorse deve essere in tempo reale.
\end{itemize}

\subsubsection{User Story 3: Monitoraggio Storico tramite GUI}

\textbf{Come} utente interessato all'analisi storica dei dati del sistema CityTwin \\
\textbf{Voglio} utilizzare la GUI \\
\textbf{In modo da} visualizzare un grafico che rappresenti il numero di nodi Mainstay e Resource online nel tempo, basato sui dati rilevati dal servizio di persistenza.

\textbf{Criteri di Accettazione:}
\begin{itemize}
  \item La GUI deve presentare un grafico temporale del numero di nodi online nel corso del tempo.
  \item L'asse delle ascisse deve rappresentare il tempo, mentre l'asse delle ordinate deve rappresentare il numero di nodi online.
  \item L'utente deve poter selezionare intervalli temporali specifici per l'analisi.
\end{itemize}

\subsection{Ubiquitous Language}

Durante lo stesso incontro, è stato stabilito un linguaggio comune (Ubiquitous Language) basato sulle specifiche del dominio del progetto CityTwin. Questo linguaggio è stato adottato da tutti i membri del team per garantire una comunicazione chiara e univoca.

\section{Secondo Project Scoping Meeting}

\subsection{User Stories}

In una sessione successiva, il team ha affinato ulteriormente le user stories. Questo passo è stato intrapreso per tenere conto di eventuali nuove informazioni emerse durante lo sviluppo iniziale del progetto, mirando a ottenere una comprensione più approfondita delle esigenze degli utenti.

\subsubsection{User Story 4: Monitoraggio dello Stato del fiume tramite GUI}

\textbf{Come} utente interessato al monitoraggio dello stato dei fiumi \\
\textbf{Voglio} utilizzare la GUI \\
\textbf{In modo da} visualizzare chiaramente lo stato attuale del River Monitor (Safe, Warned, Evacuating).

\textbf{Criteri di Accettazione:}
\begin{itemize}
  \item La GUI deve presentare lo stato corrente del River Monitor in modo evidente.
  \item Lo stato deve essere visivamente distintivo per una rapida interpretazione.
\end{itemize}

\subsubsection{User Story 5: Gestione dello Stato del River tramite GUI}

\textbf{Come} utente autorizzato alla gestione degli stati del River Monitor \\
\textbf{Voglio} utilizzare la GUI \\
\textbf{In modo da}  poter passare manualmente dallo stato di "Warned" a "Evacuating" e da "Evacuating" a "Safe".

\textbf{Criteri di Accettazione:}
\begin{itemize}
  \item La GUI deve fornire pulsanti chiaramente identificabili per passare tra gli stati.
  \item L'utente autorizzato deve poter effettuare la transizione di stato senza difficoltà.
\end{itemize}

\subsubsection{User Story 6: Aggiunta/Rimozione Nodi Resource in Tempo Reale}

\textbf{Come} amministratore del sistema CityTwin \\
\textbf{Voglio} avere la possibilità di aggiungere o rimuovere nuovi nodi Resource \\
\textbf{In modo da} adattare il sistema alle mutevoli esigenze delle smart city.

\textbf{Criteri di Accettazione:}
\begin{itemize}
  \item L'amministratore deve poter aggiungere nuovi nodi Resource senza interrompere il funzionamento del sistema.
  \item La rimozione di nodi Resource non deve causare problemi di integrità del sistema.
  \item Le modifiche devono essere effettive in tempo reale.
\end{itemize}

\subsection{Ubiquitous Language}

Durante lo stesso incontro, il linguaggio comune è stato rivisto e aggiornato per riflettere meglio le dinamiche emergenti durante il processo di sviluppo. Questo assicura una comunicazione continua e chiara tra i membri del team.

\subsection{Requirement Breakdown Structure}

Successivamente, il team ha proceduto con la scomposizione dei requisiti in una struttura gerarchica. Questa attività, completata anch'essa nel primo incontro di scoping, ha permesso di ottenere una visione dettagliata delle funzionalità e delle caratteristiche del sistema.

\section{Terzo Project Scoping Meeting}

\subsection{Requirement Breakdown Structure}

L'ultima fase di dettaglio della struttura dei requisiti, anch'essa completata nel terzo incontro, si è concentrata su eventuali modifiche o aggiunte necessarie per riflettere completamente le esigenze del progetto CityTwin.

\section{Project Overview Statement}

\subsection{Problemi}

In un'analisi approfondita, sono stati identificati e documentati i problemi chiave che il team affronterà durante lo sviluppo del sistema CityTwin.

\subsection{Goal del Progetto}

Il team ha delineato il fine ultimo del progetto CityTwin, stabilendo il risultato desiderato e l'obiettivo a lungo termine.

\subsection{Obiettivi del Progetto}

Sono stati elencati gli obiettivi specifici del progetto, fornendo una guida chiara sulle tappe da raggiungere per raggiungere il goal finale.

\subsection{Criteri di Successo}

Nella stessa fase, sono stati definiti i criteri che indicheranno il successo del progetto, offrendo un punto di riferimento oggettivo per valutare il raggiungimento degli obiettivi.

\subsection{Assunzioni, Rischi e Ostacoli}

Infine, sono state documentate le assunzioni fatte, i rischi identificati e gli ostacoli potenziali che potrebbero influire sul corso del progetto, preparando il team a gestire queste variabili in corso d'opera.
