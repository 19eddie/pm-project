\chapter{Planning}

Prima di esplorare le sessioni di planning, è essenziale identificare i partecipanti chiave coinvolte in ciascuna riunione. Le figure seguenti parteciperanno attivamente alle sessioni di planning:

\begin{itemize}
    \item Il project manager
    \item Un architetto software
    \item Il team di sviluppo: membri multidisciplinari, inclusi sviluppatori, tester e designer
    \item Un esperto in IoT e sensoristica
    \item Un esperto in sicurezza informatica
\end{itemize}

\section{Prima Joint Project Planning Session}

\subsection{Agenda}

Il focus primario del primo incontro consiste nell'elaborare la Work Breakdown Structure (WBS) in relazione alle funzionalità identificate nella Struttura di Scomposizione delle Responsabilità (RBS). La fase di prioritizzazione dei requisiti ha richiamato la partecipazione di tutti i membri presenti durante la riunione. Durante questo processo, si è cercato di determinare l'importanza relativa dei vari requisiti, coinvolgendo attivamente ciascun partecipante nella discussione e nell'assegnazione di priorità alle diverse componenti.. L'agenda includerà:

\begin{itemize}
    \item Sviluppo e definizione della Work Breakdown Structure (WBS) sulla base dei requisiti definiti precedentemente.
    \item Discussione sulle dipendenze tra le attività.
\end{itemize}

\subsection{Deliverables}

Il gruppo di sviluppo, basandosi sulla Struttura della WBS elaborata durante la riunione, ha proceduto a stimare la durata di ciascuna attività identificata. La valutazione dei tempi è stata effettuata attraverso l'applicazione della \textit{Delphi Tecnique}, in cui ogni partecipante ha espresso in modo anonimo la propria opinione sulla durata dell'attività in esame. Successivamente, tutte le valutazioni raccolte sono state presentate al team di sviluppo per favorire una discussione approfondita sulle diverse proposte. Questo processo è stato seguito da iterazioni aggiuntive, che si sono svolte con le stesse modalità descritte. La fase di stima si conclude quando si giunge a un consenso sulla durata o, più frequentemente, quando si raggiunge il numero massimo di iterazioni prestabilito. In quest'ultimo caso, la durata assegnata è la media pesata delle stime raccolte nell'ultima iterazione.

\section{Seconda Joint Project Planning Session}

\subsection{Agenda}

\subsection{Deliverables}

\section{Terza Joint Project Planning Session}

\subsection{Agenda}

\subsection{Deliverables}
