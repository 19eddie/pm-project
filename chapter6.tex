\chapter{Closing}

\section{Accettazione da Parte del Committente}
Il progetto ha ottenuto l'accettazione formale da parte del committente in linea con i criteri di accettazione definiti durante la fase di scoping. La Work Breakdown Structure (WBS) dettagliata è stata utilizzata come riferimento per garantire il soddisfacimento dei requisiti stabiliti. Tutti i deliverable sono stati presentati al committente e il processo di accettazione è stato formalizzato attraverso la firma di documenti di accettazione.

\section{Deployment della Soluzione}
Il deployment della soluzione CityTwin ha rappresentato un passaggio cruciale nel processo di implementazione del sistema di Digital Twin all'interno della città di Roma. La complessità del progetto ha richiesto una pianificazione dettagliata e una stretta collaborazione tra il team di sviluppo di TwinLink Solutions e gli stakeholder della città.

Il deployment ha coinvolto diverse fasi:

\subsection{Installazione dei Nodi Mainstay}

I nodi Mainstay, che costituiscono la struttura portante del sistema, sono stati installati in posizioni strategiche all'interno della città. La loro installazione è stata pianificata per massimizzare la copertura e l'efficacia del sistema.

\subsection{Installazione dei Nodi Resource}

I nodi Resource, che rappresentano sensori, attuatori o entità complesse, sono stati installati in diverse aree della città per rilevare informazioni cruciali. Ad esempio, sensori per la rilevazione della qualità dell'aria, dell'acqua e dell'inquinamento acustico sono stati posizionati in punti strategici per monitorare costantemente le condizioni ambientali. Gli attuatori, invece, sono stati installati per rispondere in tempo reale alle informazioni ricevute dai nodi Mainstay, ad esempio agendo in caso di emergenze come incendi o alluvioni.

\subsection{Integrazione e Sincronizzazione dei Nodi}

Una volta installati, i nodi Mainstay e Resource sono stati integrati nel sistema e sincronizzati tra loro per garantire la coerenza e l'affidabilità dei dati. Questa fase ha richiesto un'attenta configurazione e test per assicurare il corretto funzionamento dell'intero sistema.

\section{Documentazione del Progetto}
Durante tutte le fasi di sviluppo e implementazione del progetto CityTwin, è stata posta particolare attenzione alla creazione di documentazione esaustiva. Questa documentazione è stata redatta con l'obiettivo di garantire una chiara comprensione del sistema da parte di tutti gli stakeholder coinvolti e di agevolare eventuali processi futuri di manutenzione e aggiornamento.

Il cuore della documentazione consiste in un manuale utente completo, il quale fornisce istruzioni dettagliate sull'utilizzo del sistema CityTwin. Attraverso questo manuale, gli utenti possono apprendere come accedere alla piattaforma, visualizzare i dati, interagire con il sistema e risolvere eventuali problemi comuni.

In aggiunta al manuale utente, è stata preparata una documentazione tecnica dettagliata. Questo documento descrive l'architettura del sistema, le tecnologie impiegate, le decisioni di progettazione e i dettagli implementativi delle funzionalità. Rivolto principalmente agli sviluppatori e agli amministratori di sistema, questo documento permette loro di comprendere a fondo il funzionamento interno del sistema.

Parallelamente, sono stati redatti documenti che illustrano i vari processi utilizzati durante lo sviluppo del progetto. Questa documentazione include i processi di sviluppo, la gestione dei cambiamenti, la gestione dei rischi e la gestione della qualità, fornendo una guida chiara su come affrontare le varie attività durante il ciclo di vita del progetto.

Un'altra parte importante della documentazione riguarda i test eseguiti durante lo sviluppo del sistema. Questi documenti descrivono in dettaglio i piani di test, gli scenari di test e i risultati ottenuti, contribuendo a garantire la qualità complessiva del sistema e ad identificare eventuali difetti da correggere.

Infine, è stata redatta una lista completa delle risorse impiegate nel progetto, compresi hardware, software e risorse umane. Questo documento aiuta a tenere traccia delle risorse utilizzate e a pianificare eventuali necessità future.

Tutta la documentazione è stata organizzata in modo chiaro e accessibile, utilizzando formati standard e strumenti di gestione documentale. È stata resa disponibile a tutti gli stakeholder rilevanti e sarà mantenuta e aggiornata nel tempo per riflettere eventuali modifiche o evoluzioni del sistema.

\section{Audit Post-implementazione}
L'audit post-implementazione è stato condotto per valutare il successo del progetto CityTwin e identificare eventuali aree di miglioramento. Sono state considerate le seguenti domande:
\begin{itemize}
    \item \textbf{Gli obiettivi del progetto sono stati raggiunti?}\\ \textit{Sì, il funzionamento del sistema finale rispetta gli obiettivi prefissati.}
    \item \textbf{Il progetto è stato completato rispettando i limiti di tempo, budget e specifiche?}\\ \textit{Sì, il progetto è stato consegnato entro la data di scadenza prevista, rispettando il budget e soddisfacendo le specifiche concordate.}
    \item \textbf{Il committente è soddisfatto del risultato del progetto?}\\ \textit{Sì, il committente ha espresso soddisfazione per il risultato finale del progetto e ha confermato che soddisfa le sue esigenze e aspettative.}
    \item \textbf{Il business value previsto si è concretizzato?}\\ \textit{Sì, il progetto ha fornito valore aggiunto alla città di Roma, migliorando l'efficienza e la sostenibilità attraverso l'implementazione del sistema CityTwin.}
    \item \textbf{Che lezione è stata imparata relativamente alla metodologia di gestione del progetto scelta?}\\ \textit{Sono state identificate alcune aree in cui la metodologia di gestione del progetto potrebbe essere migliorata, inclusa una maggiore comunicazione e coordinamento tra i membri del team.}
    \item \textbf{Come ha seguito la metodologia il team?}\\ \textit{Il team ha seguito la metodologia di gestione del progetto in modo efficace, rispettando i tempi e i processi stabiliti e affrontando con successo le sfide incontrate durante l'implementazione del progetto.}
\end{itemize}

L'audit post-implementazione è stato condotto in modo approfondito, coinvolgendo tutte le parti interessate e utilizzando dati concreti e feedback qualitativi per valutare il successo del progetto. I risultati dell'audit sono stati utilizzati per redigere il rapporto finale e identificare eventuali azioni correttive o migliorative future.

\section{Rapporto Finale}
